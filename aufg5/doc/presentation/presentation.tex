%%%
% Compile this with ConTeXt/texexec; it is no LaTeX! 
\enableregime[utf8]

%%% D E F %%%

\definepapersize[scrsamsung][width=222mm, height=125mm]
\definecolor[TUBlue][r=.0429688, g=.164062, b=.316406]
\definecolor[TULightBlue][r=.386719, g=.421875, b=.554688]
\newdimen\BgThickBarHeight
\newdimen\BgThinBarHeight
\BgThinBarHeight=5pt


%%% C O N F I G %%%

\setuppapersize[scrsamsung][scrsamsung]
\setuplayout
	[ height=\dimexpr(.95\paperheight)
	, width=\dimexpr(.95\paperwidth)
	, backspace=.025\paperwidth
	, topspace=.025\paperheight
	, header=1.2em
	, headerdistance=2em
	, footer=1.2em
	, footerdistance=2em
	, margin=0pt
	]

\setupcolors[state=start]

\BgThickBarHeight=\dimexpr(\topspace+\headerheight+.25\headerdistance)
\definelayer[slideBGlayer]

\defineoverlay[slideBG][{\setlayer[slideBGlayer][x=0sp, y=0sp]
{\blackrule[width=\paperwidth, height=\BgThickBarHeight, color=TUBlue]}
\setlayer[slideBGlayer][x=0sp, y=\BgThickBarHeight]
{\blackrule[width=\paperwidth, height=\BgThinBarHeight, color=TULightBlue]}
%\setlayer[slideBGlayer][x=0pt, y=\dimexpr(\paperheight-\BgThickBarHeight)]
%{\blackrule[width=\paperwidth, height=\BgThickBarHeight, color=TUBlue]}
\flushlayer[slideBGlayer]}]

\setupbackgrounds[paper][background=slideBG]

\setupheader[color=white, style=\ss\bf]
\setupheadertexts[]
\setupheadertexts[Regressionstest für Java2WSDL][Software-Technologie 2, WS 2014/2015]

\setupfooter[color=TULightBlue, style=\ss]
\setupfootertexts[Felix Kluge, Marcel, Javier Sagastuy, Alejandro Escalante, André Arnold]
\setupfootertexts[{\currentdate[year,--,mm,--,dd]}][\pagenumber/\totalnumberofpages]

\setupitemize[1,packed]


%%% T E X T %%%

\starttext
\ss\TUBlue
%\showframe

\section{Ablauf}
\startitemize[n]
\item Explorative Vortests: Wie funktioniert die Bibliothek
\item Planung von Tests
\item Implementation des nötigen Testrahmens und nicht XML-bezogener Tests
\item Implementation der Daten und Tests für XML-bezogenen Tests
\stopitemize

\section{Vorgehensmethoden}
\startitemize
\item Programmierung nach Testdesign, nicht nach Diffs
\stopitemize

\section{Hindernisse}
\startitemize
\item Keiner vertraut mit Java-Werkzeugen, Probleme bei Konfiguration
\item Fehlende Anforderungsspezifikationen zur Bibliothek
\item Fehlende Komponententests: Testbare Aspekte mussten selbst erschlossen werden
\item Programmatisch unbehebbarer Fehler beim Manipulieren der Dateinamen
\stopitemize

\page[yes]
\section{Testbare Aspekte der Interfaces}
\startcolumns[2] \tfx
\startitemize
\item (x) Methodenanzahl
	\startitemize
	\item (x) 0 [NoMethod]
	\item (x) 1 [OneMethod] 
 	\item (x) viele (identisch Signatur als letzte Fall) [ManyMethods]
 	\stopitemize
 	
\item (x) Methodensignatur:
	\startitemize
	\item (x) Returntypen [ReturnTypes]
	\item (x) Parametertypen [ParameterTypes]
	\item (x) Anzahl von Parametern [ParameterNumber]
	\item (--) Methodensichtbarkeit (public, protected, private) (Nur public in Interface erlaubt)
	\item (x) Methodengültigkeitsbereich (Instanz oder Klasse; static oder non-static) [StaticMethods]
	\item (x) Exceptions [ExceptionMethods]
	\item (--) Überschreibbarkeit (final)
	\item (--) Annotationen (@...)
	\stopitemize

\item (x) eigene Typen [CustomClasses]
	\startitemize
	\item (x) als Param
	\item (x) als Return
	\stopitemize

\item (x) Methodenparameter [IdentifierCharacters]
	\startitemize
	\item (x) Bezeichner:
		\startitemize
		\item (x) Möglichen Schriftzeichen
		\stopitemize
	\stopitemize
\stopitemize
\stopcolumns

\section{Fazit}
\startitemize
\item Entwicklung mit JUnit und XMLUnit lief gut
\item hat Spaß gemacht und war lehrreich
\stopitemize

\stoptext
