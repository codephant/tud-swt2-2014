%%%
% Compile this with ConTeXt/texexec; it is no LaTeX! 
\enableregime[utf8]

%%% Configuration %%%

\setuppapersize[S6][S6]


\setupheadertexts[{\inframed[offset=overlay, frame=off, bottomframe=on]{
Software-Technologie 2, WS 2014/2015
--- Regressionstest für Java2WSDL
}}]

\setupfootertexts[{\inframed[frame=off, topframe=on]{
\currentdate[year,--,mm,--,dd]
--- Felix Kluge, Marcel, Javier Sagastuy, Alejandro Escalante, Andre Arnold
--- \pagenumber/\totalnumberofpages
}}]

\setupitemize[1,packed]


%%% Marco definitions %%%

\def\taskmark#1%
{\framed[strut=yes,offset=overlay]{#1}}

\def\taskopen{\taskmark ·}
\def\taskdone{\taskmark x}
\def\taskskip{\taskmark --}
 

\starttext

\section{Ablauf}
\startitemize[n]
\item Explorative Vortests: Wie funktioniert die Bibliothek
\item Planung von Tests
\item Implementation des nötigen Testrahmens und nicht XML-bezogener Tests
\item Implementation der Daten und Tests für XML-bezogenen Tests
\stopitemize

\section{Vorgehensmethoden}
\startitemize
\item Programmierung nach Testdesign, nicht nach Diffs
\stopitemize

\section{Hindernisse}
\startitemize
\item Keiner vertraut mit Java-Werkzeugen, Probleme bei Konfiguration
\item Fehlende Anforderungsspezifikationen zur Bibliothek
\item Fehlende Komponententests: Testbare Aspekte mussten selbst erschlossen werden
\item Programmatisch unbehebbarer Fehler beim Manipulieren der Dateinamen
\stopitemize

\page[yes]
\section{Testbare Aspekte der Interfaces}
\start\tfx
\startitemize
\item (x) Methodenanzahl
	\startitemize
	\item (x) 0 [NoMethod]
	\item (x) 1 [OneMethod] 
 	\item (x) viele (identisch Signatur als letzte Fall) [ManyMethods]
 	\stopitemize
 	
\item (x) Methodensignatur:
	\startitemize
	\item (x) Returntypen [ReturnTypes]
	\item (x) Parametertypen [ParameterTypes]
	\item (x) Anzahl von Parametern [ParameterNumber]
	\item (--) Methodensichtbarkeit (public, protected, private) (Nur public in Interface erlaubt)
	\item (x) Methodengültigkeitsbereich (Instanz oder Klasse; static oder non-static) [StaticMethods]
	\item (x) Exceptions [ExceptionMethods]
	\item (--) Überschreibbarkeit (final)
	\item (--) Annotationen (@...)
	\stopitemize

\item (x) eigene Typen [CustomClasses]
	\startitemize
	\item (x) als Param
	\item (x) als Return
	\stopitemize

\item (x) Methodenparameter [IdentifierCharacters]
	\startitemize
	\item (x) Bezeichner:
		\startitemize
		\item (x) Möglichen Schriftzeichen: https://stackoverflow.com/questions/11774099
		\stopitemize
	\stopitemize
\stopitemize
\stop%\tfx -- smaller font

\section{Fazit}
\startitemize
\item Entwicklung mit JUnit und XMLUnit lief gut
\item hat Spaß gemacht und war lehrreich
\stopitemize

\stoptext
